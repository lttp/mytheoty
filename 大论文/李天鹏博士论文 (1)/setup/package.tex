% !Mode:: "TeX:UTF-8"
%  Authors: 张井   Jing Zhang: prayever@gmail.com     天津大学2010级管理与经济学部信息管理与信息系统专业硕士生
%           余蓝涛 Lantao Yu: lantaoyu1991@gmail.com  天津大学2008级精密仪器与光电子工程学院测控技术与仪器专业本科生


%%%%%%%%%%%%%%%%%%%%%%%%----------@wchlong
\usepackage[table]{xcolor}% wchlong 表格颜色,必须放在\usepackage{tikz}前,否则冲突

\usepackage{threeparttable}%%%表格注释用

\usepackage{tikz}
\usetikzlibrary{arrows,positioning,automata,decorations,fit,backgrounds,calc,shapes,snakes}
\usepgflibrary{shapes.geometric} % LATEX and plain TEX and pure pgf
\usetikzlibrary{shapes.geometric} % LATEX and plain TEX when using Tik Z
\usepgflibrary{decorations.pathmorphing} % LATEX and plain TEX and pure pgf
\usetikzlibrary{decorations.pathmorphing} % LATEX
\usetikzlibrary{arrows}
\usetikzlibrary{decorations.markings}
\usetikzlibrary{shapes}
\usetikzlibrary{plotmarks}
\usepackage{pgfplots}
\pgfplotsset{compat=1.7}
\usepackage{epstopdf}% 定义axis
\usepackage{longtable}
\usepackage{threeparttable}
%\usepackage{ulem}%------------下划线波浪
%\usepackage[titletoc]{appendix}
\usepackage{longtable}

\usepackage{url}

%%%%%%%%%%%%%%%%%%%%%%%%%%%%
%%%%%%%%%% Package %%%%%%%%%%%%
\usepackage{graphicx}                       % 支持插图处理
\usepackage[a4paper,text={150true mm,224true mm},top=35.5true mm,left=30true mm,head=5true mm,headsep=2.5true mm,foot=8.5true mm]{geometry}
                                            % 支持版面尺寸设置

\usepackage{titlesec}                       % 控制标题的宏包
\usepackage{titletoc}                       % 控制目录的宏包
\usepackage{fancyhdr}                       % fancyhdr宏包 支持页眉和页脚的相关定义
\usepackage[fontset=windowsnew,UTF8]{ctex}                     % 支持中文显示
\usepackage{color}                          % 支持彩色
\usepackage{amsmath}                        % AMSLaTeX宏包 用来排出更加漂亮的公式
\usepackage{amssymb}                        % 数学符号生成命令
\usepackage[below]{placeins}                %允许上一个section的浮动图形出现在下一个section的开始部分,还提供\FloatBarrier命令,使所有未处理的浮动图形立即被处理
\usepackage{flafter}                        % 使得所有浮动体不能被放置在其浮动环境之前,以免浮动体在引述它的文本之前出现.
\usepackage{multirow}                       % 使用Multirow宏包,使得表格可以合并多个row格
\usepackage{booktabs}                       % 表格,横的粗线;\specialrule{1pt}{0pt}{0pt}
\usepackage{longtable}                      % 支持跨页的表格。
\usepackage{tabularx}                       % 自动设置表格的列宽
\usepackage{rotating}
\usepackage{diagbox}
\usepackage{subfigure}                      % 支持子图 %centerlast 设置最后一行是否居中
\usepackage[subfigure]{ccaption}            % 支持子图的中文标题
\usepackage[sort&compress,numbers]{natbib}  % 支持引用缩写的宏包
\usepackage{enumitem}                       % 使用enumitem宏包,改变列表项的格式
\usepackage{calc}                           % 长度可以用+ - * / 进行计算
\usepackage{txfonts}                        % 字体宏包
\usepackage{bm}                             % 处理数学公式中的黑斜体的宏包
\usepackage[amsmath,thmmarks,hyperref]{ntheorem}  % 定理类环境宏包,其中 amsmath 选项用来兼容 AMS LaTeX 的宏包
\usepackage{CJKnumb}                        % 提供将阿拉伯数字转换成中文数字的命令
\usepackage{indentfirst}                    % 首行缩进宏包
\usepackage{CJKutf8}                        % 用在UTF8编码环境下,它可以自动调用CJK,同时针对UTF8编码作了设置。
%\usepackage{hypbmsec}                      % 用来控制书签中标题显示内容
%\usepackage{enumerate}
\usepackage{mathrsfs}




\usepackage{algorithmic}%----
\usepackage{algorithm}
% \usepackage{algpseudocode} 
%\usepackage[ruled,vlined]{algorithm2e}
%\usepackage[noend,ruled,algochapter]{algorithm2e} %wchlong
%\usepackage{algorithm}
%\usepackage[linesnumbered,boxed]{algorithm2e} %有边框
%\usepackage[linesnumbered]{algorithm2e}% 没有边框
%\usepackage{algpseudocode}
% \usepackage[ruled,vlined]{algorithm2e}
% 生成有书签的 pdf 及其生成方式。通常可以在 tjumain.tex 文件的第一行选择 pdflatex 或者是 dvipdfmx 编译手段。如果选择前者,则使用 pdflatex + pdflatex 编译; 如果选择后者,在编译的时候选择 latex + bibtex + latex + latex 编译。出现混淆的时候,系统会报错。

% 如果您的pdf制作中文书签有乱码使用如下命令,就可以解决了
\def\atemp{dvipdfmx}\ifx\atemp\usewhat
\usepackage[dvipdfmx,unicode,               % dvipdfmx 编译, 加入了中文复制,粘贴支持引擎。
            pdfstartview=FitH,
            bookmarksnumbered=true,
            bookmarksopen=true,
            colorlinks=false,
            pdfborder={0 0 1},
            citecolor=blue,
            linkcolor=red,
            anchorcolor=green,
            urlcolor=blue,
            breaklinks=true
            ]{hyperref}
\fi

\def\atemp{pdflatex}\ifx\atemp\usewhat
\usepackage{cmap}                            % pdflatex 编译时,可以生成可复制、粘贴的中文 PDF 文档, 缺点是在Windows上显示时效果不大好,字体发虚
\usepackage[pdftex,unicode,
            %CJKbookmarks=true,
            bookmarksnumbered=true,
            bookmarksopen=true,
            colorlinks=false,
            pdfborder={0 0 1},
            citecolor=blue,
            linkcolor=red,
            anchorcolor=green,
            urlcolor=blue,
            breaklinks=true
            ]{hyperref}
\fi

